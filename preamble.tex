\usepackage[utf8]{inputenc}
\usepackage[T1]{fontenc}
\usepackage[polish]{babel}

\usepackage[colorlinks=true,linkcolor=black,urlcolor=blue]{hyperref}
\usepackage{bookmark}
\usepackage[dvipsnames]{xcolor}
\usepackage{float}
\usepackage[shortlabels]{enumitem}
\usepackage{parskip}

\usepackage{amsmath,amsfonts,amssymb,amsthm}
\usepackage{tikz}
\usetikzlibrary{positioning}
\usepackage{listings}
\lstdefinestyle{cppstyle}{ 
   language=C++, 
   basicstyle=\ttfamily\small,  % Use a monospaced font 
   keywordstyle=\color{blue},    % Keywords in blue 
   commentstyle=\color{green!50!black},    % Comments in green 
   stringstyle=\color{red},      % Strings in red 
   numbers=left,                 % Line numbers on the left 
   numberstyle=\tiny\color{gray}, % Style for line numbers 
   stepnumber=1,                 % Number every line 
   numbersep=5pt,                % Distance between line numbers and code 
   frame=single,                 % Frame around the code 
   backgroundcolor=\color{black!5}, % Background color 
   showstringspaces=false,       % Don't show spaces in strings 
   tabsize=2                     % Tab size 
}
\lstset{
literate=%
{ą}{{\k{a}}}1
{Ą}{{\k{A}}}1
{ć}{{\'c}}1
{Ć}{{\'{C}}}1
{ę}{{\k{e}}}1
{Ę}{{\k{E}}}1
{ł}{{\l{}}}1
{Ł}{{\L{}}}1
{ń}{{\'n}}1
{Ń}{{\'N}}1
{ó}{{\'o}}1
{Ó}{{\'O}}1
{ś}{{\'s}}1
{Ś}{{\'S}}1
{ż}{{\.z}}1
{Ż}{{\.Z}}1
{ź}{{\'z}}1
{Ź}{{\'Z}}1
}
\usepackage{algpseudocode}

\usepackage[margin=1in]{geometry}
\pagestyle{empty}

\usepackage{import}
\usepackage{xifthen}
\usepackage{pdfpages}
\usepackage{transparent}

\newcommand{\incfig}[2]{
    \ifthenelse{\isempty{#2}}{
        \def\svgwidth{0.7\columnwidth}
    }{
        \def\svgwidth{#2\columnwidth}
    }
    \import{./figures/}{#1.pdf_tex}
}

% theorems
\usepackage{thmtools}
\usepackage[framemethod=TikZ]{mdframed}
\mdfsetup{skipabove=1em,skipbelow=0em}

\theoremstyle{definition}

\declaretheoremstyle[
    headfont=\bfseries\sffamily\color{ForestGreen!70!black}, bodyfont=\normalfont,
    mdframed={
        linewidth=2pt,
        rightline=false, topline=false, bottomline=false,
        linecolor=ForestGreen, backgroundcolor=ForestGreen!5,
    }
]{thmgreenbox}

\declaretheoremstyle[
    headfont=\bfseries\sffamily\color{NavyBlue!70!black}, bodyfont=\normalfont,
    mdframed={
        linewidth=2pt,
        rightline=false, topline=false, bottomline=false,
        linecolor=NavyBlue, backgroundcolor=NavyBlue!5,
    }
]{thmbluebox}

\declaretheoremstyle[
    headfont=\bfseries\sffamily\color{NavyBlue!70!black}, bodyfont=\normalfont,
    mdframed={
        linewidth=2pt,
        rightline=false, topline=false, bottomline=false,
        linecolor=NavyBlue
    }
]{thmblueline}

\declaretheoremstyle[
    headfont=\bfseries\sffamily\color{RawSienna!70!black}, bodyfont=\normalfont,
    mdframed={
        linewidth=2pt,
        rightline=false, topline=false, bottomline=false,
        linecolor=RawSienna, backgroundcolor=RawSienna!5,
    }
]{thmredbox}

\declaretheoremstyle[
    headfont=\bfseries\sffamily\color{RawSienna!70!black}, bodyfont=\normalfont,
    numbered=no,
    mdframed={
        linewidth=2pt,
        rightline=false, topline=false, bottomline=false,
        linecolor=RawSienna, backgroundcolor=RawSienna!1,
    },
    qed=\qedsymbol
]{thmproofbox}

\declaretheoremstyle[
    headfont=\bfseries\sffamily\color{NavyBlue!70!black}, bodyfont=\normalfont,
    numbered=no,
    mdframed={
        linewidth=2pt,
        rightline=false, topline=false, bottomline=false,
        linecolor=NavyBlue, backgroundcolor=NavyBlue!1,
    },
]{thmexplanationbox}

\declaretheorem[style=thmgreenbox, name=Definicja]{definition}
\declaretheorem[style=thmbluebox, numbered=no, name=Przykład]{eg}
\declaretheorem[style=thmbluebox, numbered=no, name=Pomysł]{idea}
\declaretheorem[style=thmredbox, name=Propozycja]{prop}
\declaretheorem[style=thmredbox, name=Twierdzenie]{theorem}
\declaretheorem[style=thmredbox, name=Algorytm]{{algorithm}}
\declaretheorem[style=thmredbox, name=Lemat]{lemma}
\declaretheorem[style=thmredbox, numbered=no, name=Wniosek]{corollary}
\declaretheorem[style=thmblueline, numbered=no, name=Uwaga]{remark}
\declaretheorem[style=thmblueline, numbered=no, name=Notacja]{notation}
\declaretheorem[style=thmproofbox, name=Dowód]{replacementproof}
\declaretheorem[style=thmexplanationbox, name=Dowód]{tmpexplanation}
\declaretheorem[style=thmblueline, name=Zadanie]{task}

\renewenvironment{proof}[1][\proofname]{\vspace{-10pt}\begin{replacementproof}}{\end{replacementproof}}
\newenvironment{explanation}[1][]{\vspace{-10pt}\begin{tmpexplanation}}{\end{tmpexplanation}}

\newcommand\N{\ensuremath{\mathbb{N}}}
\newcommand\R{\ensuremath{\mathbb{R}}}
\newcommand\Z{\ensuremath{\mathbb{Z}}}
\renewcommand\O{\ensuremath{\emptyset}}
\newcommand\Q{\ensuremath{\mathbb{Q}}}
\newcommand\C{\ensuremath{\mathbb{C}}}
\newcommand{\D}[1]{\mathop{\mathrm{d}#1}}
\DeclareMathOperator{\ins}{int}
\newcommand{\eval}[1]{\left. #1 \right\rvert}


% Add \contra symbol to denote contradiction
\usepackage{stmaryrd} % for \lightning
\newcommand\contra{\scalebox{1.5}{$\lightning$}}

% lecture stuff
\makeatletter
\def\@lecture{}%
\newcommand{\lecture}[3]{
    \ifthenelse{\isempty{#3}}{%
        \def\@lecture{Zajęcia #1}%
    }{%
        \def\@lecture{Zajęcia #1: #3}%
    }%
    \section*{\@lecture}
    \addcontentsline{toc}{section}{\@lecture}
    \marginpar{\small\textsf{\mbox{#2}}}
    \medskip
}

\usepackage{fancyhdr,lastpage,authoraftertitle}
\pagestyle{fancy}
\fancyhf{}
\fancyhead[R]{\MyAuthor}
\fancyhead[L]{\MyTitle}
\fancyfoot[R]{Strona \thepage/\pageref{LastPage}}
\fancyfoot[L]{\@lecture}
\makeatother
\title{Tytuł}
\author{Maciej Mikołajczak}
\renewcommand{\headrulewidth}{1pt}
\renewcommand{\headruleskip}{2pt}
\setlength{\headheight}{14pt}
\renewcommand{\footrulewidth}{1pt}
\renewcommand{\arraystretch}{1.2}
\renewcommand{\thempfootnote}{\arabic{mpfootnote}}
